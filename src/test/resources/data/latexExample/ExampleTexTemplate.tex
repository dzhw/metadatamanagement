\documentclass[a4paper]{article}
 
\usepackage[urlbordercolor={1 1 1}]{hyperref}
\usepackage{graphicx}
\usepackage{fancyhdr}
\usepackage[utf8]{inputenc}
\usepackage[margin=1in,headsep=.7in]{geometry}
\usepackage[ngerman]{babel}
\usepackage{multicol}
\usepackage{hyperref}
\usepackage{ulem}
\usepackage{pgfplots}
\usepackage{array}
\usepackage{tocloft}

% Space between subsection number and subsection title for the command \subsection{}
\setlength{\cftsubsecnumwidth}{4em}

% PGF Plots
\pgfplotsset{compat=1.8}
\usepgfplotslibrary{statistics}
\setlength{\columnsep}{1cm}

% Define Pagestyle
\pagestyle{fancy}
\geometry{a4paper,left=28mm,right=28mm, top=32mm, bottom=22mm} 

% Table Columns equivalent to l,c,r but with wraping the text
\newcolumntype{L}[1]{>{\raggedright\let\newline\\\arraybackslash\hspace{0pt}}m{#1}}
\newcolumntype{C}[1]{>{\centering\let\newline\\\arraybackslash\hspace{0pt}}m{#1}}
\newcolumntype{R}[1]{>{\raggedleft\let\newline\\\arraybackslash\hspace{0pt}}m{#1}}

%Definition of author und title
\hypersetup{
	pdfauthor={fdz.DZHW}
	pdfkeywords={Variable Report}
	pdftitle={Variable Report}
}
  
\begin{document}

%INTRODUCTION
\include{Introduction}

%VARIABLES, EVERY VARIABLE HAS A OWN PAGE
\newpage
\section{Variablen}
<#list dataSet.variableIds as variableId>	

	<#if variables[variableId]??>
	<#assign variable = variables[variableId]>
	\subsection{${variable.name}}	
	
	%TABLE FOR THE VARIABLE DETAILS	
	\noindent\textbf{Variable:}\\
		\begin{tabular}{ll}
			Name & ${variable.name!"-"} \\
			Label & <#if variable.label??>${variable.label.de!"-"}<#else>-</#if> \\
			Beschreibung & <#if variable.description??>${variable.description.de!"-"}<#else>-</#if> \\			
			Skalenniveau & <#if variable.scaleLevel??>${variable.scaleLevel.de!"-"}<#else>-</#if> \\					
			Zugangswege & 
			<#list variable.accessWays as accessWay> 
				${accessWay}, 
			<#else>
				Keine Zugangswege!
			</#list> \\
			Panelvariablen & <#if variable.sameVariablesInPanel??>
			<#list variable.sameVariablesInPanel as variableInPanel> 
				${variableInPanel}, 
			<#else>
				-
			</#list> 
			<#else>-
			</#if> \\	
			Eingangsfilter & <#if variable.filterDetails??>${variable.filterDetails.filterExpression!"-"}<#else>-</#if> \\	
			Erzeugungsregel & <#if variable.generationDetails??>${variable.generationDetails.rule!"-"}<#else>-</#if> \\
		\end{tabular}	
	
	<#if variable.atomicQuestionId??>
		%TABLE FOR QUESTION DETAILS
		<#assign question = questions[variable.atomicQuestionId]>
		\vspace*{1 cm}
		\noindent\textbf{Frage:}\\
		\begin{tabular}{ll}
			Name & <#if question.name??>${question.name!"-"}<#else>-</#if> \\
			Einleitung der Frage & <#if question.introduction??>${question.introduction.de!"-"}<#else>-</#if> \\
			Fragetext & <#if question.questionText??>${question.questionText.de!"-"}<#else>-</#if> \\
			Ausfülleinweisung & <#if question.instruction??>${question.instruction.de!"-"}<#else>-</#if> \\
		\end{tabular}
	</#if>

	
	<#if variable.scaleLevel.de == "ordinal" && variable.statistics??>
		%TABLE FOR THE ORDINAL STATISTICS
		\vspace*{1 cm}
		\noindent\textbf{Statistische Daten:}\\	
		\begin{tabular}{ll}
			Median & ${variable.statistics.median!"-"}
		\end{tabular}
	</#if>
	
	
	<#if variable.scaleLevel.de == "kontinuierlich" && variable.statistics?? >
		%TABLE FOR THE METRIC STATISTICS
		\vspace*{1 cm}
		\noindent\textbf{Statistische Daten:}\\	
		\begin{tabular}{ll}
			<#if variable.statistics.median??>
				Median & ${variable.statistics.median} \\
			</#if>
			<#if variable.statistics.meanValue??>
				Arithmetisches Mittel & ${variable.statistics.meanValue} \\ 
			</#if>
			<#if variable.statistics.standardDeviation??>
				Standardabweichung & ${variable.statistics.standardDeviation} \\ 
			</#if>
			<#if variable.statistics.minimum??>
				Minimum & ${variable.statistics.minimum} \\ 
			</#if>
			<#if variable.statistics.maximum??>
				Maximum & ${variable.statistics.maximum} \\ 
			</#if>
			<#if variable.statistics.skewness??>
				Schiefe & ${variable.statistics.skewness} \\ 
			</#if>
			<#if variable.statistics.kurtosis??>
				Wölbung & ${variable.statistics.kurtosis} \\ 
			</#if>				
			<#if variable.statistics.firstQuartile??>
				Unteres Quartil & ${variable.statistics.firstQuartile} \\ 
			</#if>
			<#if variable.statistics.thirdQuartile??>
				Oberes Quartil & ${variable.statistics.thirdQuartile} \\ 
			</#if>
		\end{tabular}
	</#if>
			
	<#if variable.values??>
	<#assign isAMssingCounter = isNotAMissingCounterMap[variable.id]>
	<#if variable.scaleLevel.de == "ordinal" || variable.scaleLevel.de == "nominal">
		%TABLE FOR THE NOMINAL / ORDINAL VALUES	
		\vspace*{1 cm}
		\noindent\textbf{Werte:}\\
		\begin{table}[!h]
			\centering
			\begin{tabular}{|L{1.7cm}|c|L{4.5cm}|c|c|C{1.5cm}|}
				\hline
				& \textbf{Code} & \textbf{Wertelabel} & \textbf{Häufigkeiten} & \textbf{Prozent} & \textbf{gültige Prozent} \\
				\hline
				\hline  				
				\textbf{Gültige Werte} & & & & &\\
				
				<#if isAMssingCounter <= 20>
					<#list variable.values as value>  
						<#if !value.isAMissing>					
							& ${value.code!"-"} & ${value.label.de!"-"} & ${value.absoluteFrequency!"-"} & ${value.relativeFrequency!"-"} & ${value.validRelativeFrequency!"-"} \\
						</#if>
					<#else>
						- \\
					</#list>
				<#else>
					<#list firstTenIsNotAMissingValues[variable.id] as value> 
						& ${value.code!"-"} & ${value.label.de!"-"} & ${value.absoluteFrequency!"-"} & ${value.relativeFrequency!"-"} & ${value.validRelativeFrequency!"-"} \\
					</#list>
					& ... & ... & ... & ... & ... \\
					<#list lastTenIsNotAMissingValues[variable.id] as value>  
						& ${value.code!"-"} & ${value.label.de!"-"} & ${value.absoluteFrequency!"-"} & ${value.relativeFrequency!"-"} & ${value.validRelativeFrequency!"-"} \\
					</#list>
				</#if>				
				
				
				\hline
				\textbf{Fehlende Werte} & & & & &\\
				<#list variable.values as value>  
					<#if value.isAMissing>					
						& ${value.code!"-"} & ${value.label.de!"-"} & ${value.absoluteFrequency!"-"} & ${value.relativeFrequency!"-"} & - \\
					</#if>
				<#else>
					- \\
				</#list>
				\hline
				\textbf{Summe (gültig)} & & & \textbf{${variable.valueSummary.totalValidAbsoluteFrequency!"-"}} & \textbf{${variable.valueSummary.totalValidRelativeFrequency!"-"}} & \textbf{100}\\
				\textbf{Summe (gesamt)} & & & \textbf{${variable.valueSummary.totalAbsoluteFrequency!"-"}} & \textbf{100} & \textbf{-}\\				
				\hline
			\end{tabular}
			\caption{Werte der Variable ${variable.name}}
		\end{table}		
	</#if>	
		
	<#if variable.scaleLevel.de == "kontinuierlich">
		%TABLE FOR THE CONTINOUS VALUES	
		\vspace*{1 cm}
		\noindent\textbf{Werte:}\\
		\begin{table}[!h]
			\centering
			\begin{tabular}{|L{1.7cm}|c|L{4.5cm}|c|c|}
				\hline
				& \textbf{Code} & \textbf{Wertelabel} & \textbf{Häufigkeiten} & \textbf{Prozent} \\
				\hline
				\hline  				
				\textbf{Gültige Werte} & & & &\\
				<#list variable.values as value>  
					<#if !value.isAMissing && value.code??>					
						& ${value.code!"-"} & ${value.label.de!"-"} & ${value.absoluteFrequency!"-"} & ${value.relativeFrequency!"-"} \\
					</#if>
				<#else>
					- \\
				</#list>
				
				\hline
				\textbf{Fehlende Werte} & & & &\\
				<#list variable.values as value>  
					<#if value.isAMissing>					
						& ${value.code!"-"} & ${value.label.de!"-"} & ${value.absoluteFrequency!"-"} & ${value.relativeFrequency!"-"}  \\
					</#if>
				<#else>
					- \\
				</#list>
				\hline
				\textbf{Summe (gesamt)} & & & \textbf{${variable.valueSummary.totalAbsoluteFrequency!"-"}} & \textbf{100} \\				
				\hline
			\end{tabular}
			\caption{Werte der Variable ${variable.name}}
		\end{table}		
	</#if>	
	<#else>
	Die Variable ${variable.id} hat keine Werte.
	</#if>
	
	<#if variable.statistics.median?? && variable.statistics.firstQuartile?? && variable.statistics.thirdQuartile?? && variable.statistics.highWhisker?? && variable.statistics.lowWhisker??>
		\begin{figure}[!h]
		\center			
		\begin{tikzpicture}
		\begin{axis}
    		[
    			/pgf/number format/.cd,
	        use comma,
	        1000 sep={.},
    			ytick = {${variable.statistics.lowWhisker?c?replace(",",".")},${variable.statistics.firstQuartile?c},${variable.statistics.median?c},${variable.statistics.thirdQuartile?c},${variable.statistics.highWhisker?c}},
    			boxplot/draw direction=y,
  			boxplot/every box/.style={fill=gray!50},
  			boxplot/every median/.style={ultra thick, red},
    			width=0.2\textwidth,
	      	height=0.35\textwidth,
    			hide x axis,
    			axis y line*=left
    		]
    		\addplot+[
    			boxplot prepared={
      		median=${variable.statistics.median?c},
      		upper quartile=${variable.statistics.thirdQuartile?c},
      		lower quartile=${variable.statistics.firstQuartile?c},
    	  		upper whisker=${variable.statistics.highWhisker?c},
    		  	lower whisker=${variable.statistics.lowWhisker?c?replace(",",".")}
    			},
    		] coordinates {};
  		\end{axis}
		\end{tikzpicture}
		\caption{Boxplot der Variable ${variable.name}}
		\end{figure}
	</#if>
	
	\newpage
	<#else>
	Die Variable ${variableId} wird im Datensatz aufgeführt, jedoch liegt diese nicht im System vor.
	</#if>
<#else>
	Keine Variablen!
</#list>

\end{document}